%==============================================================================
% Sjabloon onderzoeksvoorstel bachproef
%==============================================================================
% Gebaseerd op document class `hogent-article'
% zie <https://github.com/HoGentTIN/latex-hogent-article>

% Voor een voorstel in het Engels: voeg de documentclass-optie [english] toe.
% Let op: kan enkel na toestemming van de bachelorproefcoördinator!
\documentclass{hogent-article}

% Beperk (waarschuwingen voor verwaarloosbare) over- en underfull-instanties
\hfuzz=5.0pt
\hbadness=10000
\setlength{\emergencystretch}{3em}

% Invoegen bibliografiebestand
\addbibresource{voorstel.bib}

% Informatie over de opleiding, het vak en soort opdracht
\studyprogramme{Professionele bachelor toegepaste informatica}
\course{Bachelorproef}
\assignmenttype{Onderzoeksvoorstel}
% Voor een voorstel in het Engels, haal de volgende 3 regels uit commentaar
% \studyprogramme{Bachelor of applied information technology}
% \course{Bachelor thesis}
% \assignmenttype{Research proposal}

\academicyear{2025-2026} % TODO: pas het academiejaar aan

% TODO: Werktitel
\title{Omzetten van UML-diagrammen (PlantUML) naar "configuration management"-code (Ansible)\\\textemdash\space onderzoek \& proof-of-concept}

% TODO: Studentnaam en emailadres invullen
\author{Lucas Ludueña-Segre}
\email{lucas.luduenasegre@student.hogent.be}

% TODO: Medestudent
% Gaat het om een bachelorproef in samenwerking met een student in een andere
% opleiding? Geef dan de naam en emailadres hier
% \author{Yasmine Alaoui (naam opleiding)}
% \email{yasmine.alaoui@student.hogent.be}

% Link naar GitHub-repository
\projectrepo{https://github.com/lucasluduenasegre/latex-hogent-bachproef-nl-25-26-luduenasegrelucas}

% TODO: Geef de co-promotor op
\supervisor[Co-promotor]{Y. Rousseaux (District09, \href{mailto:Yoeri.Rousseaux@district09.gent}{Yoeri.Rousseaux@district09.gent})}

% Binnen welke specialisatierichting uit 3TI situeert dit onderzoek zich?
% Kies uit deze lijst:
%
% - Mobile \& Enterprise development
% - AI \& Data Engineering
% - Functional \& Business Analysis
% - System \& Network Administrator
% - Mainframe Expert
% - Als het onderzoek niet past binnen een van deze domeinen specifieer je deze
%   zelf
%
\specialisation{System \& Network Administrator}
\keywords{UML, Unified Modeling Language, PlantUML, MermaidJS, configuration management, Ansible, Puppet, Chef, compilers, transpilers}

\begin{document}

\begin{abstract}
  % Hier schrijf je de samenvatting van je voorstel, als een doorlopende tekst van één paragraaf. Let op: dit is geen inleiding, maar een samenvattende tekst van heel je voorstel met inleiding (voorstelling, kaderen thema), probleemstelling en centrale onderzoeksvraag, onderzoeksdoelstelling (wat zie je als het concrete resultaat van je bachelorproef?), voorgestelde methodologie, verwachte resultaten en meerwaarde van dit onderzoek (wat heeft de doelgroep aan het resultaat?).
  Op HOGENT, binnen de opleiding Toegepaste Informatica en meerbepaald het keuzetraject ``System \& Network Administrator'', worden er OLODs (opleidingsonderdelen) gegeven die aan de hand van labo's bijdragen aan het leren van serverinfrastructuren creeëren en configureren. Deze infrastructuren worden geïmplementeerd aan de hand van netwerk- en implementatiediagrammen, ontworpen met UML-modeling software.
  De probleemstelling voor deze bachelorproef vertrekt vanuit het foutgevoelige en tijdrovende proces waarbij (complexe) UML-diagrammen eerst geïnterpreteerd en daarna manueel vertaald worden naar een configureerbare infrastructuur.
  De hoofdonderzoeksvraag is hoe men, voor een potentiële serverinfrastructuur, netwerk- en implementatiediagrammen (meerbepaald in tekstuele vorm via PlantUML/MermaidJS) met zo weinig mogelijk manuele interventie kan omzetten naar een overeenkomende, gebruiksklare omgeving ondersteund door ``configuration management''-code (zoals Ansible, Puppet of Chef).
  De doelstelling van dit onderzoek is om een proof-of-concept transpiler te ontwikkelen die gegeven netwerk- en implementatiediagrammen als invoer rechtstreeks kan vertalen naar ``configuration management''-code als uitvoer. 
  De geplande testomgeving zal bestaan uit 2 tot 3 reeds uitgerolde (virtuele) servers (al dan niet via Infrastructure-as-Code tools), en zullen bestaan uit een web-, DNS- en monitoringserver (zonder nog eventuele uitbreidingen).
  Qua methodologie wordt er eerst een studie naar transpilers uitgevoerd, parallel met een evaluatie van te gebruiken tools (PlantUML, MermaidJS, Ansible, Puppet en Chef). Vervolgens wordt de testomgeving ontworpen (via UML-diagrammen), opgezet en gedocumenteert. Dan wordt de transpiler ontwikkeld en ten slotte systematisch getest tegen (onder andere) de kwaliteit en idempotentie van de uitvoer.
  Verwacht wordt dat deze oplossing een meerwaarde biedt voor docenten (en studenten): zij zouden hiermee sneller en efficiënter nieuwe labo-omgevingen kunnen ontwerpen en deze tegelijk voorzien van bijkomende ``configuration management''-code, aangezien de kloof tussen de ontwerp- en implementatiefase zo wordt verkleint.

  % TODO!!

\end{abstract}

\tableofcontents

% De hoofdtekst van het voorstel zit in een apart bestand, zodat het makkelijk
% kan opgenomen worden in de bijlagen van de bachelorproef zelf.
%---------- Inleiding ---------------------------------------------------------

% TODO: Is dit voorstel gebaseerd op een paper van Research Methods die je
% vorig jaar hebt ingediend? Heb je daarbij eventueel samengewerkt met een
% andere student?
% Zo ja, haal dan de tekst hieronder uit commentaar en pas aan.

%\paragraph{Opmerking}

% Dit voorstel is gebaseerd op het onderzoeksvoorstel dat werd geschreven in het
% kader van het vak Research Methods dat ik (vorig/dit) academiejaar heb
% uitgewerkt (met medestudent VOORNAAM NAAM als mede-auteur).
% 

\section{Inleiding}%
\label{sec:inleiding}

% Waarover zal je bachelorproef gaan? Introduceer het thema en zorg dat volgende zaken zeker duidelijk aanwezig zijn:

% \begin{itemize}
%   \item kaderen thema
%   \item de doelgroep
%   \item de probleemstelling en (centrale) onderzoeksvraag
%   \item de onderzoeksdoelstelling
% \end{itemize}

% Denk er aan: een typische bachelorproef is \textit{toegepast onderzoek}, wat betekent dat je start vanuit een concrete probleemsituatie in bedrijfscontext, een \textbf{casus}. Het is belangrijk om je onderwerp goed af te bakenen: je gaat voor die \textit{ene specifieke probleemsituatie} op zoek naar een goede oplossing, op basis van de huidige kennis in het vakgebied.

% De doelgroep moet ook concreet en duidelijk zijn, dus geen algemene of vaag gedefinieerde groepen zoals \emph{bedrijven}, \emph{developers}, \emph{Vlamingen}, enz. Je richt je in elk geval op it-professionals, een bachelorproef is geen populariserende tekst. Eén specifiek bedrijf (die te maken hebben met een concrete probleemsituatie) is dus beter dan \emph{bedrijven} in het algemeen.

% Formuleer duidelijk de onderzoeksvraag! De begeleiders lezen nog steeds te veel voorstellen waarin we geen onderzoeksvraag terugvinden.

% Schrijf ook iets over de doelstelling. Wat zie je als het concrete eindresultaat van je onderzoek, naast de uitgeschreven scriptie? Is het een proof-of-concept, een rapport met aanbevelingen, \ldots Met welk eindresultaat kan je je bachelorproef als een succes beschouwen?

Binnen de opleiding Toegepaste Informatica op HOGENT hebben de meeste OLODs (opleidingsonderdelen) gelinkt aan het keuzetraject ``System \& Network Administrator'' de opzet om de creatie en het beheer van (virtuele) serveromgevingen aan te leren. Bij het opzetten van dit soort infrastructuur, is helderheid over elke stap van het proces en communicatie met elke -- al dan niet technisch ingestelde -- stakeholder cruciaal.
Om deze duidelijkheid te waarborgen, wordt er binnen de opleiding (meerbepaald tijdens de analyse-OLODs) ook aangeleerd om te werken met UML-modeling software zoals Visual Paradigm\footnote{\url{https://www.visual-paradigm.com/}} en draw.io\footnote{\url{https://www.drawio.com/}}. Deze software wordt gebruikt om diagrammen te ontwerpen voor softwareontwikkeling, zoals Activity Diagrams, System Sequence Diagrams, domeinmodellen en BPMN-diagrammen.

Men kan met deze software ook netwerk- en implementatiediagrammen ontwerpen, die de specificaties, software-elementen (databanken, webservers, firewalls, \ldots) en verbanden tussen verschillende servers in een bepaalde infrastructuur verduidelijken. Deze servers worden op basis van deze diagrammen daarna respectievelijk opgezet via ``Infrastructure-as-Code''-tools (zoals Terraform\footnote{\url{https://developer.hashicorp.com/terraform}}, OpenTofu\footnote{\url{https://opentofu.org/}}, Vagrant\footnote{\url{https://developer.hashicorp.com/vagrant}}, \ldots) en geconfigureerd met ``configuration management''-tools (Ansible\footnote{\url{https://docs.ansible.com/}}, Puppet\footnote{\url{https://www.puppet.com/}}, Chef\footnote{\url{https://www.chef.io/}}, \ldots). De HOGENT-OLODs ``DevOps Project: Operations'', ``Cybersecurity Advanced'' en meerbepaald ``Infrastructure Automation'' introduceren ook het gebruik van Ansible, één van de leidende ``configuration management''-tools.

De \emph{concrete probleemstelling/casus} voor het onderzoek:
Het huidige proces waarbij (complexe) netwerk- en implementatiediagrammen worden geïnterpreteerd en manueel worden omgezet tot een ``configuration management''-ondersteunde serveromgeving kan, volgens \textcite{Nicacio2020}, tijdrovend en vatbaar voor ambiguïteit en foute interpretaties zijn.

Bij implementatiediagrammen kan er onder meer onduidelijkheid onstaan bij (de versies van) software-packages, hun afhankelijkheden/dependencies, de firewall-regels voor elke server en hoe dit het bereik buiten het lokale netwerk beïnvloedt.
Voor netwerkdiagrammen kan dit bijvoorbeeld gaan over hardwarespecificaties van de servers, (de versies van) hun besturingssystemen en de declaratie van de DNS server(s).

De \emph{hoofdonderzoeksvraag} voor deze bachelorproef:
Hoe kan men een gegeven netwerk- en implementatiediagram (in tekstuele vorm via bv. PlantUML\footnote{\url{https://plantuml.com/}}) van een (virtuele) serveromgeving met minimale menselijke interventie omzetten tot een overeenkomende, gebruiksklare ``configuration management''-ondersteunde (bv. met Ansible) serveromgeving?

Voor deze serveromgeving wordt ervan uit gegaan dat de eenheden (bare-metal servers, virtuele machines, containers, \ldots) alvorens uitgerold zijn via een ``Infrastructure-as-Code''-tool en dat ze enkel nog achteraf geconfigureerd moeten worden via een ``configuration management''-tool.

Enkele deelvragen met betrekking tot het \emph{probleemdomein}:

\begin{itemize}
  \item Wat zijn de doeleinden (business- of technisch-gericht) van UML?
  \item Welke infrastructuur-gerelateerde informatie kan een netwerk- en implementatiediagram \emph{niet} weergeven (dat een ``Infrastructure-as-Code''- en/of ``configuration management''-tool wel kan)?
  \item Welke stappen zijn er tussen het ontwerp van UML-diagrammen en de eerste opstelling van een ``configuration management''-ondersteunde omgeving?
\end{itemize}

Verdere deelvragen met betrekking tot het \emph{oplossingsdomein}

\begin{itemize}
  \item Naast PlantUML en MermaidJS, met welke andere talen kan men UML-diagrammen ontwerpen?
  \item Buiten Ansible, Puppet en Chef, wat zijn andere ``configuration management''-tools die voor dit soort doeleinden geschikt zijn?
  \item Aangezien er al tools bestaan om UML-diagrammen te genereren op basis van ``configuration management''-ondersteunde omgevingen, zoals één van \textcite{teramako2023}, welke meerwaarde zou deze oplossing (die in de omgekeerde richting werkt) kunnen bieden?
  \item Is het mogelijk om deze oplossing te ontwikkelen aan de hand van ``reverse engineering'' van de bovenstaande tools?
\end{itemize}

De \emph{doelstelling} van deze bachelorproef is om een proof-of-concept te ontwikkelen voor een transpiler (of omzetter) van PlantUML-diagrammen naar Ansible-code. PlantUML en Ansible zijn de twee tools die op dit moment voor ogen staan, maar er zullen tijdens het onderzoek ook alternatieven gezocht en geëvalueerd worden. 

De hoop voor deze oplossing is dat deze een meerwaarde kan bieden voor educatieve doeleinden. Neem als voorbeeld een OLOD zoals ``Cybersecurity Advanced'' binnen de opleiding Toegepaste Informatica. De tech-industrie evolueert snel en de labo's die voor dit OLOD ontworpen zijn, zullen daardoor ook snel verouderen.

Aangezien met deze oplossing de drempel tussen de ontwerp- en implementatiefase verkleint, zouden docenten sneller nieuwe (virtuele) labo-omgevingen kunnen ontwerpen, met (in theorie) kant-en-klare ``configuration management''-code om deze meteen te configureren. Voor studenten kan dit ook een kans bieden om potentiële experimenten tijdens hun studies sneller effectief vorm te geven.

%---------- Literatuurstudie | Stand van zaken ---------------------------------------------------

\section{Literatuurstudie}%
\label{sec:literatuurstudie}

% Hier beschrijf je de \emph{state-of-the-art} rondom je gekozen onderzoeksdomein, d.w.z.\ een inleidende, doorlopende tekst over het onderzoeksdomein van je bachelorproef. Je steunt daarbij heel sterk op de professionele \emph{vakliteratuur}, en niet zozeer op populariserende teksten voor een breed publiek. Wat is de huidige stand van zaken in dit domein, en wat zijn nog eventuele open vragen (die misschien de aanleiding waren tot je onderzoeksvraag!)?

% Je mag de titel van deze sectie ook aanpassen (literatuurstudie, stand van zaken, enz.). Zijn er al gelijkaardige onderzoeken gevoerd? Wat concluderen ze? Wat is het verschil met jouw onderzoek?

% Verwijs bij elke introductie van een term of bewering over het domein naar de vakliteratuur, bijvoorbeeld~\autocite{Hykes2013}! Denk zeker goed na welke werken je refereert en waarom.

% Draag zorg voor correcte literatuurverwijzingen! Een bronvermelding hoort thuis \emph{binnen} de zin waar je je op die bron baseert, dus niet er buiten! Maak meteen een verwijzing als je gebruik maakt van een bron. Doe dit dus \emph{niet} aan het einde van een lange paragraaf. Baseer nooit teveel aansluitende tekst op eenzelfde bron.

% Als je informatie over bronnen verzamelt in JabRef, zorg er dan voor dat alle nodige info aanwezig is om de bron terug te vinden (zoals uitvoerig besproken in de lessen Research Methods).

% Voor literatuurverwijzingen zijn er twee belangrijke commando's:
% \autocite{KEY} => (Auteur, jaartal) Gebruik dit als de naam van de auteur
%   geen onderdeel is van de zin.
% \textcite{KEY} => Auteur (jaartal)  Gebruik dit als de auteursnaam wel een
%   functie heeft in de zin (bv. ``Uit onderzoek door Doll \& Hill (1954) bleek
%   ...'')

% Je mag deze sectie nog verder onderverdelen in subsecties als dit de structuur van de tekst kan verduidelijken.


\subsection{Het belang van Unified Modeling Language (UML)}%
\label{subsec:belang_uml}%

``Unified Modeling Language'', of afgekort UML, is een taal/framework die binnen de softwareontwikkeling-discipline geaccepteerd is als de standaard voor het weergeven van object-georiënteerde modellen. UML wordt hoofdzakelijk gebruikt in de industrie-, business- en IT-sector, met als doel om de ontwerp- en implementatiefase te ondersteunen met diagrammen zoals netwerk- en implementatiediagrammen, maar ook Activity Diagrams, System Sequence Diagrams, domeinmodellen en BPMN-diagrammen. Deze kunnen door zowel business als technische stakeholders geïnterpreteerd worden voor hun respectievelijke doeleinden. \autocite{Koc2021}

Binnen de opleiding Toegepaste Informatica worden er twee UML-modeling programma's gebruikt; UML Visual Paradigm en draw.io. Dit zijn twee tools die werden geëvalueerd tijdens een vergelijkende studie van \textcite{Lu2023}, waarin deze respectievelijk 122 en 114 scoorden op een maximum van 165 op basis van hun eigenschappen en functies.

PlantUML en MermaidJS\footnote{\url{https://mermaid.js.org}} zijn tools waarmee men UML-diagrammen kan ontwerpen via menselijk leesbare en gemakkelijk onderhoudbare (door middel van versiebeheertools zoals Git\footnote{\url{https://git-scm.com/}}) tekstbestanden \autocite{Romeo2025}. Deze tools zijn ook open source, ergo de broncode van deze software is voor iedereen beschikbaar.

\subsection{Configuration management en haar toepassingen}%
\label{subsec:cfgmgmt_toepassingen}

``Configuration management'' heeft als principe om een (grote) set van manuele, repetitieve configuratietaken (installeren van packages, beheren van firewalls, instellen van de default gateways en DNS-servers, \ldots) automatisch te laten uitvoeren op servers. In het beste geval verloopt dit proces op een idempotente manier; het herhalen van het hetzelde proces geeft exact hetzelfde resultaat. Deze manier van werken schaalt beter voor omgevingen met honderden of duizenden servers, aangezien het aanzienlijk minder tijdrovend en vatbaar voor menselijke fouten is. \autocite{Likitha2022}

``Configuration management'' staat echter niet in voor het creeëren en opzetten van servers; dat is de taak van ``Infrastructure-as-Code'' tools zoals Terraform, OpenTofu en Vagrant.

Ansible, Puppet en Chef zijn drie open source tools die in de praktijk het meest gebruikt worden om ``configuration management'' toe te passen \autocite{Masek2018}.

\subsection{Stand van zaken binnen de OLODs in de opleiding Toegepaste Informatica}%
\label{subsec:stand_van_zaken_binnen_olods_ti}

Er zijn drie noemenswaardige OLODs, binnen het keuzetraject ``System \& Network Administrator'' van de opleiding Toegepaste Informatica, die de nadruk leggen op het opzetten en configureren van (virtuele) serveromgevingen met configuration management tools en deze visualiseren met UML-diagrammen.

\begin{itemize}
  \item ``Cybersecurity Advanced''
  \item ``DevOps Project: Operations''
  \item ``Infrastructure Automation''
\end{itemize}

Het OLOD ``Cybersecurity Advanced'' introduceert het concept van ``blue-teaming'' binnen het gebied van cybersecurity aan de hand van een reeks virtuele labo's die individueel worden uitgevoerd. Het gebruik van ``configuration management'' of UML-diagrammen is hierbij niet strikt noodzakelijk, maar het geeft de student wel een aanzienlijk voordeel aangezien deze zo de labo-omgeving reproduceerbaar en visueel overzichtelijk kan maken. Bovendien wordt de (basis-)opstelling alvast voorgesteld met een netwerkdiagram ontworpen via PlantUML.

``DevOps Project: Operations'' is een OLOD dat het tweede luik vormt van een tweeledig groepsproject, het eerste luik zijnde ``Real-life Integrated Software Engineering'' voor de studenten binnen ``Mobile \& enterprise developer''. Het doel van het ``operations-team'' in dit project, is om de backend van een webapplicatie (ontwikkeld door het ``development-team'') te ondersteunen met een productie-waardige serverinfrastructuur. Deze omgeving moet automatisch kunnen opgezet worden via zowel ``Infrastructure-as-Code''- als ``configuration management''-tools (met in het bijzonder Ansible).
Om de onderliggende werking van deze omgeving te visualiseren voor al dan niet technische stakeholders, moet er ook een netwerk- en implementatiediagram (of een combinatie ervan) ontworpen worden met UML-modeling software.

Tot slot biedt het OLOD ``Infrastructure Automation'' een reeks virtuele labo's die verschillende tools (Jenkins\footnote{\url{https://www.jenkins.io/}}, Ansible, Prometheus\footnote{\url{https://prometheus.io/}} en Kubernetes\footnote{\url{https://kubernetes.io/}}) introduceren voor het automatisch opzetten, configureren en monitoren van serveromgevingen. Nogmaals, hierbij wordt het gebruik van UML-diagrammen niet verplicht, maar aangeraden.

\subsection{Compilers, transpilers en hun doel}%
\label{subsec:overzicht_transpilers}

Compilers vertalen de broncode van een high-level, menselijk leesbare taal (zoals Java\footnote{\url{https://www.java.com/}} of C\#\footnote{\url{https://dotnet.microsoft.com/en-us/languages/csharp}}) tot equivalente instructies in een andere, vaak low-level taal of machinecode \autocite{Seidl2012}.

``Source-to-source compilers'', algemeen gekend als ``transpilers'', zijn compilers die code geschreven in één high-level (programmeer-)taal als invoer aannemen en deze omzetten naar code in een andere high-level taal als uitvoer. De use cases voor deze software kunnen zich ook uitbreiden tot het optimaliseren of omzetten van code naar een andere/nieuwere versie van eenzelfde taal. \autocite{Ilyushin2016}

%---------- Methodologie ------------------------------------------------------

\section{Methodologie}%
\label{sec:methodologie}

% Hier beschrijf je hoe je van plan bent het onderzoek te voeren. Welke onderzoekstechniek ga je toepassen om elk van je onderzoeksvragen te beantwoorden? Gebruik je hiervoor literatuurstudie, interviews met belanghebbenden (bv.~voor requirements-analyse), experimenten, simulaties, vergelijkende studie, risico-analyse, PoC, \ldots?

% Valt je onderwerp onder één van de typische soorten bachelorproeven die besproken zijn in de lessen Research Methods (bv.\ vergelijkende studie of risico-analyse)? Zorg er dan ook voor dat we duidelijk de verschillende stappen terug vinden die we verwachten in dit soort onderzoek!

% Vermijd onderzoekstechnieken die geen objectieve, meetbare resultaten kunnen opleveren. Enquêtes, bijvoorbeeld, zijn voor een bachelorproef informatica meestal \textbf{niet geschikt}. De antwoorden zijn eerder meningen dan feiten en in de praktijk blijkt het ook bijzonder moeilijk om voldoende respondenten te vinden. Studenten die een enquête willen voeren, hebben meestal ook geen goede definitie van de populatie, waardoor ook niet kan aangetoond worden dat eventuele resultaten representatief zijn.

% Uit dit onderdeel moet duidelijk naar voor komen dat je bachelorproef ook technisch voldoen\-de diepgang zal bevatten. Het zou niet kloppen als een bachelorproef informatica ook door bv.\ een student marketing zou kunnen uitgevoerd worden.

% Je beschrijft ook al welke tools (hardware, software, diensten, \ldots) je denkt hiervoor te gebruiken of te ontwikkelen.

% Probeer ook een tijdschatting te maken. Hoe lang zal je met elke fase van je onderzoek bezig zijn en wat zijn de concrete \emph{deliverables} in elke fase?

\subsection{Fasering van de bachelorproef}%
\label{subsec:fasering_bap}

Deze bachelorproef zal uitgevoerd worden in de volgende fases, van 9 februari (week 1) t.e.m. 22 mei (week 15), 2026:

\begin{itemize}
  \item{Fase 1A (week 1-7): Studie over (het ontwikkelen van) transpilers}
  \item{Fase 1B (week 1-2): Evaluatie van de bestaande tools in de betrokken vakgebieden}
  \item{Fase 2 (week 3-5): Ontwerp van de testomgeving (UML-diagrammen)}
  \item{Fase 3A (week 6-7): Implementatie van de testomgeving}
  \item{Fase 3B (week 6-7): UML-diagrammen herschrijven naar PlantUML/MermaidJS}
  \item{Fase 3C (week 6-10): Opstellen van testcases}
  \item{Fase 4 (week 8-15): Ontwikkeling van de transpiler}
  \item{Fase 5 (week 11-15): Testen van de bekomen oplossing}
\end{itemize}

Indien een fase begint met hetzelfde nummer maar eindigt met een andere letter, indiceert dit dat ze op hetzelfde moment (kunnen) beginnen, eindigen (met fases 1A, 3C en 5 als uitzondering) en parallel worden uitgevoerd.

% \subsection{GANTT-schema}%
% \label{subsec:gantt_schema}
% 
% \begin{center}
  % \begin{figure}
    % \includegraphics[width=.8\textwidth]{images/voorstel-gantt.png}
    % \caption{\label{fig:voorstel-gantt.png}GANTT-schema voor de opgelijste fases.}
  % \end{figure}
% \end{center}

\subsection{Fase 1A (week 1-7): Studie over (het ontwikkelen van) transpilers}%
\label{subsec:studie_ontwikkelen_transpilers}

Bij het begin van het onderzoek zal er parallel met de andere voorbereidende fases (1-3) een studie uitgevoerd worden over hoe transpilers werken en hoe deze ontwikkeld worden.
De deliverable voor deze fase is een simpele, maar functionele transpiler (zonder UML-diagrammen en ``configuration management''-code) die de basis zal vormen voor de toekomstige, finale versie.

\subsection{Fase 1B (week 1-2): Evaluatie van de bestaande tools in de betrokken vakgebieden}%
\label{subsec:evaluatie_bestaande_tools_betrokken_vakgebieden}

Hier zal er een afweging gemaakt worden voor de verschillende tools binnen UML-modeling en ``configuration management'' die gebruikt zullen worden tijdens de ontwikkeling van de proof-of-concept.
De te evalueren tools voor UML-modeling zijn PlantUML en MermaidJS. Voor het ``configuration management''-gedeelte zal er gekeken worden naar Ansible, Puppet en Chef. Deze opsomming van tools is echter niet-uitputtend, en andere tools kunnen nog in het vizier komen.

\subsection{Fase 2 (week 3-5): Ontwerp van de testomgeving (UML-diagrammen)}%
\label{subsec:hardware_softwarespecs_testomgeving}

In deze fase wordt de (virtuele) testomgeving voor de transpiler ontworpen aan de hand van netwerk- en implementatiediagrammen. In deze diagrammen zullen de hardware- (Opslag, RAM, CPU cores, \ldots) en softwarespecificaties (distributie, packages, \ldots) van elke server beschreven worden, alsook hoe deze met elkaar interageren (firewalls, eventuele backup-regels, \ldots). Het initiële ontwerp zal nog niet met PlantUML/MermaidJS verlopen maar eerder met GUI-ondersteunde tools zoals Visual Paradigm en draw.io.

\subsection{Fase 3A (week 6-7): Implementatie van de testomgeving}%
\label{subsec:implementatie_testomgeving}

Tijdens deze fase wordt de infrastructuur (ontworpen in fase 2) opgezet met ``Infrastructure-as-Code''-tool(s) en geconfigureerd met een ``configuration management''-tool (Ansible/Puppet/Chef). Er wordt intussen ook gedocumenteerd hoe het ``configuration management''-gedeelte wordt opgezet, want dit vormt de target output voor de transpiler. De aanpassingen in verband met ``configuration management'' worden vervolgens ongedaan gemaakt alvorens de resulterende ``configuration management''-code van de transpiler wordt uitgevoerd op de testomgeving.

\subsection{Fase 3B (week 6-7): UML-diagrammen herschrijven naar PlantUML/MermaidJS}%
\label{subsec:uml_diagrammen_herscrijven}

Nadat de netwerk- en implementatiediagrammen zijn ontworpen, zullen deze herschreven worden in PlantUML/MermaidJS. Deze diagrammen dienen een tweevoudig doel; ze kunnen binnen de originele context van hun taal als visuele UML-diagrammen worden gegenereerd, en ze vormen de bron-bestanden voor de transpiler, die deze zal omzetten tot ``configuration management''-code.

\subsection{Fase 3C (week 6-10): Opstellen van testcases}%
\label{subsec:opstellen_testcases}

Hier worden de testcases opgesteld waaraan de transpiler moet voldoen. De testen zich zullen hoofdzakelijk buigen over de kwaliteit van de uitvoer (aantal playbooks, idempotentie, grootte van codebase, \ldots) en de werking van de finale infrastructuur. Deze fase zal in de praktijk waarschijnlijk gedeeltelijk parallel verlopen met fase 4, aangezien er tijdens het ontwikkelen van de oplossing mogelijk nieuwe testcases kunnen ontstaan.

\subsection{Fase 4 (week 8-15): Ontwikkeling van de transpiler}%
\label{subsec:ontwikkeling_poc_transpiler}

Tijdens deze fase wordt de transpiler ontwikkeld. Deze zal als invoer de netwerk- en implementatiediagrammen (ontworpen met PlantUML/MermaidJS) aannemen en ze omzetten tot ``configuration management''-code (in Ansible/Puppet/Chef). Deze fase loopt tot 22 mei, 2026, aangezien dit voor de auteur de laatste dag van de stage is.

\subsection{Fase 5 (week 11-15): Testen van de bekomen oplossing}%
\label{subsec:evaluatie_bekomen_oplossing}

Ten slotte worden de opgestelde testen uit fase 3C uitgevoerd op de (gedeeltelijk) uitgewerkte transpiler. Op basis van de resultaten zal er een rapport worden opgesteld die aangeeft in welke mate deze oplossing een antwoord biedt op de hoofdonderzoeksvraag. Nogmaals, dit zal allicht parallel verlopen met fase 4.

%---------- Verwachte resultaten ----------------------------------------------

% Hier beschrijf je welke resultaten je verwacht. Als je metingen en simulaties uitvoert, kan je hier al mock-ups maken van de grafieken samen met de verwachte conclusies. Benoem zeker al je assen en de onderdelen van de grafiek die je gaat gebruiken. Dit zorgt ervoor dat je concreet weet welk soort data je moet verzamelen en hoe je die moet meten.

% Wat heeft de doelgroep van je onderzoek aan het resultaat? Op welke manier zorgt jouw bachelorproef voor een meerwaarde?

% Hier beschrijf je wat je verwacht uit je onderzoek, met de motivatie waarom. Het is \textbf{niet} erg indien uit je onderzoek andere resultaten en conclusies vloeien dan dat je hier beschrijft: het is dan juist interessant om te onderzoeken waarom jouw hypothesen niet overeenkomen met de resultaten.

\section{Verwacht resultaat, conclusie}%
\label{sec:verwachte_resultaten}

\subsection{Verwachte te gebruiken tools}%
\label{subsec:verwachte_te_gebruiken_tools}

De meest voor de hand liggende tools die voor deze oplossing geschikt lijken, zijn PlantUML voor UML-modeling en Ansible voor ``configuration management''. Dit heeft hoofdzakelijk te maken met de voorkennis van de auteur, met in het bijzonder Ansible aangezien deze binnen de opleiding Toegepaste Informatica (en de stageplaats van de auteur) de standaard vormt als toepassing voor ``configuration management''.

\subsection{Verwachte eigenschappen van de testomgeving}
\label{subsec:verwachte_eigenschappen_testomgeving}

Deze infrastructuur zal allicht redelijk simpel gehouden worden met slechts 2 tot 3 (virtuele) servers, aangezien deze bachelorproef vooral over de ontwikkeling van een proof-of-concept gaat. Dat gezegd zijnde zal er gemikt worden op het implementeren van een web-, DNS- en monitoringserver. Mogelijke uitbreidingen kunnen een databank-, secundaire DNS- en DHCP-server zijn.

\subsection{Verwachte functionaliteit voor de transpiler}%
\label{subsec:verwachte_functionaliteit_poc}

De verwachting van deze oplossing is om met de resulterende ``configuration management''-code een minimaal werkende infrastructuur te configureren die de beschrijving van de gegeven netwerk- en implementatiediagrammen zo nauw mogelijk benadert.
De transpiler zal slechts voor een (relevant) deel van de syntax tussen PlantUML en Ansible een vertaallaag kunnen vormen, aangezien de omvang voor beide tools veel groter is dan die voor dit onderzoek.  
Als invoer zullen er enkel netwerk- en implementatiediagrammen gebruikt kunnen worden en als uitvoer zal er een deelverzameling van ondersteunde ``collections'' en ``roles'' voor de Ansible playbooks gespecifiëerd worden. Nogmaals, dit wordt een proof-of-concept, dus het onderzoek moet ook aantonen waar het potentiëel tot uitbreidingen ligt, alsook wat de limitaties zijn qua informatie uit een UML-diagram die omgezet kan worden naar ``configuration management''-code en vice versa.


\printbibliography[heading=bibintoc]

\end{document}